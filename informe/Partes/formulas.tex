\section{Formulas}

\begin{enumerate}
 \item Presi\'on de un pozo $x$ en el d\'ia $t$ despu\'es de haber sido construido:
 
  $$
    P_{x,{t+1}} = \left\{
    \begin{array}{cl}
    \textit{Undefined}	& \mbox{si } t < 0 \\
    P_{bc_x} 		& \mbox{si } t = 0 \\
    P_{x,t} \ e^{-\beta_i}	& \mbox{si } t > 0 \text{ y no hay inyecci\'on en ese d\'ia para ese pozo} \\
    P_{bc_x} V_i/\text{VI} & \mbox{si } t > 0 \text{ y hay inyecci\'on en ese d\'ia para ese pozo} \\
    \end{array}\right.
    $$

 Siendo:
 
 \begin{itemize}
  \item $i$ el d\'ia \textit{real} asociado al $t$ ($t$ es el offset desde el d\'ia de la construcci\'on del pozo e $i$ el offset desde el inicio de actividad en el yacimiento). 
  \item $P_{bc_x}\in[\text{3000, 3500}]$ (en psi) presi\'on de boca de pozo (inicial) para el pozo $x$.
  \item VI $\in[1\times 10^7, 1\times 10^9]$ (en $m^3$) volumen inicial del producto en el yacimiento
  \item V$_i$ es el volumen total del producto en el reservorio que va quedando luego de $i$ d\'ias de explotaci\'on:
  
  \begin{equation*}
    V_i = \text{VI} - \text{VE}_i + \text{VR}_i
  \end{equation*}

  
  \item VE$_i$ volumen total extra\'ido en todo el yacimiento al d\'ia $i$, i.e, la suma del potencial de los pozos para cada d\'ia que se extrae:
  
  \begin{equation*}
   VE_i = \sum_{x,t} V_{x,t}
  \end{equation*}

  
  \item VR$_i$ volumen total reinyectado en todo el yacimiento al d\'ia $i$. (Ver que VR$_i<$VE$_i$)
  \item La constante $\beta_i$ esta definida como:
  \begin{equation*}
   \beta_i = \frac{V_i}{10\ \text{VI}\ (\text{NP}_i)^{2/3}}
  \end{equation*}
  \item NP$_i$ cantidad de pozos extrayendo en el d\'ia $i$
 \end{itemize}

 \item Potencial del pozo $x$ en el d\'ia $t$ despu\'es de haber sido creado:
 
 \begin{equation*}
  V_{x,t} = \alpha_1 \frac{P_{x,t}}{\text{NP}_i} + \alpha_2 \left(\frac{P_{x,t}}{\text{NP}_i}\right)^2
 \end{equation*}

 Siendo: 
 
 \begin{itemize}
  \item $\alpha_1\in[\text{0.1, 0.6}]$ (en $m^3/$psi) una constante que es par\'ametro del simulador
  \item $\alpha_2\in[\text{0.005, 0.01}]$ (en $m^3/$psi$^2$) una constante que es par\'ametro del simulador
 \end{itemize}

\end{enumerate}
