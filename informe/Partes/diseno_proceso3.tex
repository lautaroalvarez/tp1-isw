\subsection{Ejecución de un criterio de parada}

\par A continuación vamos a mostrar el diagrama de secuencias de la ejecución del criterio de parada que es el que determina la finalización de la simulación. Va a estar compuesto por una estrategia de parada y un parámetro representativo de la estrategia (por ejemplo, una estrategia de parada podría ser "hasta que el yacimiento tenga cierto nivel de presión de los pozos" y el parámetro representativo sería ese valor de presión). Este criterio se ejecutará cada día (luego de haber ejecutado los criterios de construcción, de extracción, etc) de manera de indicarle al sistema de ejecución de criterios si debe finalizar o no la simulación (devuelve true en dicho caso). A continuación describiremos el diagrama de secuencia correspondiente a este criterio.


\subsubsection{Escenario}
\par En este caso nos enfocaremos en un criterio que determina la finalizaci�n de la simulación un vez que hayan pasado 150 días. El sistema se encuentra simulando el día 10/12/2017 y la fecha de inicio fue el 01/09/2017.

\subsubsection{Funcionamiento General}
\par Como vemos en el diagrama de secuencias de la figura \ref{fig:dia_sec_parada_1_1}, el sistema de ejecución de criterios le envía un mensaje al criterio para que se ejecute pasándole una referencia al sistema de simulación, ya que es él quien conoce los objetos fechaDeSimulación y fechaDeInicio. Este criterio le envía el mensaje ejecutar a la estrategia elegida por el usuario, que en este caso es de tipo EstrategiaDeParadaPorCantidadDeDias. Luego, la estrategia le consulta al Simulador qué fecha está simulando y cuál es la fecha de inicio. Internamente hace la resta entre las dos fechas, y le responde false a elCriterioDeParada una vez que el resultado fue menor a 150 días.

\begin{landscape}
  \begin{figure}[ht]
  \centering
    \begin{sequencediagram}
      \newinst{e1}{\shortstack{elSistema\\DeEjecucion\\DeCriterios :\\SistemaDeEjecucion\\DeCriterios}}
      \newinst{e2}{\shortstack{elCriterio\\DeParada :\\CriterioDe\\Parada}}
      \newinst{e3}{\shortstack{laEstrategia\\DeParada :\\Estrategia\\DeParada\\PorCantidad\\DeDias}}
      \newinst{e4}{\shortstack{elSimulador :\\SistemaDe\\Simulacion}}

      \postlevel
      \postlevel
      \postlevel
      \begin{call}{}{\shortstack{\textbf{ejecutar:}\\elCriterioDe\\Parada}}{e1}{true}
        \begin{call}{e1}{\shortstack{\textbf{ejecutarEn:}\\elSimulador}}{e2}{true}
          \begin{call}{e2}{\shortstack{\textbf{ejecutarEn:}\\elSimulador}}{e3}{true}
            \begin{call}{e3}{\shortstack{\texbf{fechaDe\\Simulacion}}}{e4}{10/12/2017}
            \postlevel
            \end{call}
            \postlevel
            \begin{call}{e3}{\shortstack{\texbf{fechaDe\\Inicio}}}{e4}{10/12/2017}
            
            \postlevel
            \end{call}
          \end{call}
        \end{call}
      \end{call}
    \end{sequencediagram}
    \caption{Diagrama de secuencias de la ejecuci�n de un criterio de parada.}
    \label{fig:dia_sec_parada_1_1}
  \end{figure}
\end{landscape}
