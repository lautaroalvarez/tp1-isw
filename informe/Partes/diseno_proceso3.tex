\subsection{Ejecuci�n de un criterio de parada}

\par A continuaci�n vamos a mostrar el diagrama de secuencias de la ejecuci�n del criterio de parada que es el que determina la finalizaci�n de la simulaci�n. Va a estar compuesto por una estrategia de parada y un par�metro representativo de la estrategia (por ejemplo, una estrategia de parada podr�a ser "hasta que el yacimiento tenga cierto nivel de presi�n de los pozos" y el par�metro representativo ser�a ese valor de presi�n). Este criterio se ejecutar� cada d�a (luego de haber ejecutado los criterios de construcci�n, de extracci�n, etc) de manera de indicarle al sistema de ejecuci�n de criterios si debe finalizar o no la simulaci�n (devuelve true en dicho caso). A continuaci�n describiremos el diagrama de secuencia correspondiente a este criterio.


\subsubsection{Escenario}
\par En este caso nos enfocaremos en un criterio que determina la finalizaci�n de la simulaci�n un vez que hayan pasado 150 d�as. El sistema se encuentra simulando el d�a 10/12/2017 y la fecha de inicio fue el 01/09/2017.

\subsubsection{Funcionamiento General}
\par Como vemos en el diagrama de secuencias de la figura \ref{fig:dia_sec_parada_1_1}, el sistema de ejecuci�n de criterios le env�a un mensaje al criterio para que se ejecute pas�ndole una referencia al sistema de simulaci�n, ya que es �l quien conoce los objetos fechaDeSimulaci�n y fechaDeInicio. Este criterio le env�a el mensaje ejecutar a la estrategia elegida por el usuario, que en este caso es de tipo EstrategiaDeParadaPorCantidadDeDias. Luego, la estrategia le consulta al Simulador qu� fecha est� simulando y cu�l es la fecha de inicio. Internamente hace la resta entre las dos fechas, y le responde false a elCriterioDeParada una vez que el resultado fue menor a 150 d�as.

\begin{landscape}
  \begin{figure}[ht]
  \centering
    \begin{sequencediagram}
      \newthread{entorno}{\shortstack{Entorno}}
      \newinst{e1}{\shortstack{elSistema\\DeEjecucion\\DeCriterios :\\SistemaDeEjecucion\\DeCriterios}}
      \newinst{e2}{\shortstack{elCriterio\\DeParada :\\CriterioDe\\Parada}}
      \newinst{e3}{\shortstack{laEstrategia\\DeParada :\\Estrategia\\DeParada\\PorCantidad\\DeDias}}
      \newinst{e4}{\shortstack{elSimulador :\\SistemaDe\\Simulacion}}

      \postlevel
      \postlevel
      \postlevel
      \begin{call}{entorno}{\shortstack{\textbf{ejecutar:}\\elCriterioDe\\Parada}}{e1}{}
        \begin{call}{e1}{\shortstack{\textbf{ejecutarEn:}\\elSimulador}}{e2}{true}
          \begin{call}{e2}{\shortstack{\textbf{ejecutarEn:}\\elSimulador}}{e3}{true}
            \begin{call}{e3}{\shortstack{\texbf{fechaDe\\Simulacion}}}{e4}{10/12/2017}
            \postlevel
            \end{call}
            \postlevel
            \begin{call}{e3}{\shortstack{\texbf{fechaDe\\Inicio}}}{e4}{10/12/2017}
            
            \postlevel
            \end{call}
          \end{call}
        \end{call}
      \end{call}
    \end{sequencediagram}
    \caption{Diagrama de secuencias de la ejecuci�n de un criterio de parada.}
    \label{fig:dia_sec_parada_1_1}
  \end{figure}
\end{landscape}
